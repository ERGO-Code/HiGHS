\documentclass{article}
\usepackage{amsmath}
\usepackage{multirow}

% page size
\setlength{\hoffset}{-1in}
\setlength{\voffset}{-1in}
\setlength{\oddsidemargin}{3.5cm}
\setlength{\evensidemargin}{3.5cm}
\setlength{\topmargin}{2cm}
\setlength{\textwidth}{14cm}
\setlength{\textheight}{22cm}

\newcommand{\ct}{\texttt}
\newcommand{\NULL}{\ct{NULL}}
\newcommand{\rvec}[1]{\left(#1\right)}
\newcommand{\param}[4]{
  \noindent\begin{minipage}{\textwidth}
    \textbf{#1}\\
    Type: \ct{#2}\\
    Default: \ct{#3}\\
    #4
  \end{minipage}
  \vskip 1\baselineskip
}
\newcommand{\info}[2]{
  \noindent\begin{minipage}{\textwidth}
    \textbf{#1}\\
    #2
  \end{minipage}
  \vskip 1\baselineskip
}

\newcommand{\obj}{\ct{obj}}
\newcommand{\rhs}{\ct{rhs}}
\newcommand{\lb}{\ct{lb}}
\newcommand{\ub}{\ct{ub}}
\newcommand{\numconstr}{\ct{num\_constr}}
\newcommand{\numvar}{\ct{num\_var}}
\newcommand{\x}{\ct{x}}
\newcommand{\slack}{\ct{slack}}
\newcommand{\xl}{\ct{xl}}
\newcommand{\xu}{\ct{xu}}
\newcommand{\y}{\ct{y}}
\newcommand{\z}{\ct{z}}
\newcommand{\zl}{\ct{zl}}
\newcommand{\zu}{\ct{zu}}
\newcommand{\vbasis}{\ct{vbasis}}
\newcommand{\cbasis}{\ct{cbasis}}
\newcommand{\basic}{\ct{IPX\_basic}}
\newcommand{\nonbasic}{\ct{IPX\_nonbasic}}
\newcommand{\nonbasiclb}{\ct{IPX\_nonbasic\_lb}}
\newcommand{\nonbasicub}{\ct{IPX\_nonbasic\_ub}}
\newcommand{\superbasic}{\ct{IPX\_superbasic}}

\title{IPX Reference Guide}

\begin{document}
\maketitle
\tableofcontents

%-------------------------------------------------------------------------------
\section{Functionality}
%-------------------------------------------------------------------------------

\subsection{Problem Formulation}
IPX solves linear programming (LP) problems in the form
\begin{subequations}
  \label{eq:lp}
  \begin{alignat}{2}
    \underset{\x}{\text{minimize}} &\quad& &\obj^T\x \\
    \label{eq:ax=b}
    \text{subject to} && &A\x\{\ge,\le,=\}\rhs, \\
    && &\lb\le\x\le\ub.\
  \end{alignat}
\end{subequations}
The matrix $A$ has \numconstr\ rows and \numvar\ columns. Associated with
\eqref{eq:ax=b} are dual variables \y\ with the sign convention that
\begin{subequations}
  \label{eq:signy}
  \begin{alignat}{2}
    \y[i]\ge0 &\quad& &\text{if constraint is of type $\ge$,} \\
    \y[i]\le0 && &\text{if constraint is of type $\le$,} \\
    \y[i]\text{ free} && &\text{if constraint is of type $=$.}
  \end{alignat}
\end{subequations}
Associated with $\lb\le\x$ and $\x\le\ub$ are dual variable $\zl\ge0$ and
$\zu\ge0$ respectively. Entries of $-\lb$ and $\ub$ can be infinity, in which
case the dual is fixed at zero.

\subsection{Interior Point Method}
The interior point method (IPM) computes a primal-dual point
$\rvec{\x,\slack,\xl,\xu,\y,\zl,\zu}$ that approximately satisfies
\begin{subequations}
  \label{eq:res}
  \begin{align}
    \label{eq:pres}
    &A\x+\slack=\rhs, \quad \x-\xl=\lb, \quad \x+\xu=\ub,\\
    \label{eq:dres}
    &A^T\y+\zl-\zu=\obj,
  \end{align}
\end{subequations}
and that is guaranteed to satisfy $\xl\ge0$, $\xu\ge0$, \eqref{eq:signy} and
\begin{subequations}
  \label{eq:signslack}
  \begin{alignat}{2}
    \slack[i]\le0 &\quad& &\text{if constraint is of type $\ge$,} \\
    \slack[i]\ge0 && &\text{if constraint is of type $\le$,} \\
    \slack[i]=0 && &\text{if constraint is of type $=$.}
  \end{alignat}
\end{subequations}
In theory, the IPM iterates will in the limit satisfy \eqref{eq:pres} and
\eqref{eq:dres}, and the primal objective will equal the dual objective
\begin{equation}
  \label{eq:dobj}
  \rhs^T\y + \lb^T\zl - \ub^T\zu.
\end{equation}
(Entries for which -\lb\ or \ub\ is infinity are understood to be dropped from
the sum.)

\subsection{Crossover}
The crossover method recovers an optimal basis from the interior solution. A
basis is defined by variable and constraint statuses
\begin{align}
  \vbasis[j] &\in \{\basic, \nonbasiclb, \nonbasicub, \superbasic\}, \\
  \cbasis[i] &\in \{\basic, \nonbasic\}.
\end{align}
The columns of $A$ for which $\vbasis[j]=\basic$ and the columns of the identity
matrix for which $\cbasis[i]=\basic$ form a square, nonsingular matrix of
dimension \numconstr. The corresponding basic solution $\rvec{\x,\slack,\y,\z}$
is obtained by setting
\begin{subequations}
  \begin{alignat}{3}
    \z[j]&=0& &\quad& &\text{if $\vbasis[j]=\basic$}, \\
    \x[j]&=\lb[j]& && &\text{if $\vbasis[j]=\nonbasiclb$}, \\
    \x[j]&=\ub[j]& && &\text{if $\vbasis[j]=\nonbasicub$}, \\
    \x[j]&=0& &&      &\text{if $\vbasis[j]=\superbasic$}, \\
    \y[i]&=0& &&      &\text{if $\cbasis[i]=\basic$}, \\
    \slack[i]&=0& &&  &\text{if $\cbasis[i]=\nonbasic$}
  \end{alignat}
\end{subequations}
and computing the remaining components such that $A\x+\slack=\rhs$ and
$A^T\y+\z=\obj$. The basis is primal feasible if $\lb\le\x\le\ub$ and
\eqref{eq:signslack} hold; the basis is dual feasible if \eqref{eq:signy} holds
and
\begin{subequations}
  \begin{alignat}{2}
    \z[j]\ge0 &\quad& &\text{if $\vbasis[j]=\nonbasiclb$}, \\
    \z[j]\le0 &\quad& &\text{if $\vbasis[j]=\nonbasicub$}, \\
    \z[j]=0 &\quad& &\text{if $\vbasis[j]=\superbasic$}.
  \end{alignat}
\end{subequations}
The IPX crossover consists of a primal and dual push phase. Depending on the
accuracy of the interior solution and the numerical stability of the LP problem,
the obtained basis may not be primal and/or dual feasible. In this case
reoptimization with an external simplex code is required.

%-------------------------------------------------------------------------------
\section{Usage}
%-------------------------------------------------------------------------------

\subsection{C++ Interface}
User code written in C++ must include the header file \ct{src/lp\_solver.h}.
Both the \ct{src/} and \ct{include/} directories must be in the compiler's
search path for header files.

The following code snippet illustrates the use of the C++ interface. The
complete example program can be found in \ct{example/afiro.cc} and can be
compiled by calling \ct{make} in the \ct{example/} directory.

\newpage
\begin{verbatim}
#include <cmath>
#include <iostream>
#include "lp_solver.h"

using Int = ipxint;

constexpr Int num_var = 12;
constexpr Int num_constr = 9;
const double obj[] = { -0.2194, 0.0, 0.0, 0.0, 0.0, 0.0, 0.0, 0.0, -0.32,
                       -0.5564, 0.6, -0.48 };
const double lb[num_var] = { 0.0 };
const double ub[] = { 80.0, 283.303, 283.303, 312.813, 349.187, INFINITY,
                      INFINITY, INFINITY, 57.201, 500.0, 500.501, 357.501};
// Constraint matrix in CSC format with 0-based indexing.
const Int Ap[] = { 0, 2, 6, 10, 14, 18, 20, 22, 24, 26, 28, 30, 32 };
const Int Ai[] = { 0, 5, ... };
const double Ax[] = { -1.0, 0.301, ... };
const double rhs[] = { 0.0, 80.0, 0.0, 0.0, 0.0, 0.0, 0.0, 44.0, 300.0 };
const char constr_type[] = { '<', '<', '=', '<', '<', '=', '<', '<', '<' };

int main() {
    ipx::LpSolver lps;

    // Solve the LP.
    Int status = lps.Solve(num_var, obj, lb, ub, num_constr, Ap, Ai, Ax, rhs,
                           constr_type);
    if (status != IPX_STATUS_solved) {
        // no solution
        // (invalid input, time/iter limit, numerical failure, out of memory)
        std::cout << " status: " << status << ','
                  << " errflag: " << lps.GetInfo().errflag << '\n';
        return 1;
    }

    // Get solver and solution information.
    ipx::Info info = lps.GetInfo();

    // Get the interior solution (available if IPM was started).
    double x[num_var], xl[num_var], xu[num_var], slack[num_constr];
    double y[num_constr], zl[num_var], zu[num_var];
    lps.GetInteriorSolution(x, xl, xu, slack, y, zl, zu);

    // Get the basic solution (available if crossover terminated without error).
    double xbasic[num_var], sbasic[num_constr];
    double ybasic[num_constr], zbasic[num_var];
    Int cbasis[num_constr], vbasis[num_var];
    lps.GetBasicSolution(xbasic, sbasic, ybasic, zbasic, cbasis, vbasis);

    return 0;
}
\end{verbatim}

\newpage
All classes and functions belonging to IPX are declared in namespace \ct{ipx}.
There are three classes that are intended for direct use from external code:
\ct{ipx::LpSolver}, \ct{ipx::Parameters} and \ct{ipx::Info}. The \ct{Parameters}
and \ct{Info} classes have no methods (except for a constructor that initializes
them to default values). They are used to pass control parameters to IPX and to
return information about the execution of the solver. They are documented in
Sections \ref{sec:parameters} and \ref{sec:info}.

The \ct{LpSolver} class provides the user interface to the LP solver. Its
methods are documented through source code comments in \ct{src/lp\_solver.h}.
The method
\begin{verbatim}
    Int Solve(Int num_var, const double* obj, const double* lb,
              const double* ub, Int num_constr, const Int* Ap, const Int* Ai,
              const double* Ax, const double* rhs, const char* constr_type);
\end{verbatim}
requires the LP problem in the form \eqref{eq:lp} with the constraint matrix $A$
given in compressed sparse column (CSC) format. That means, \ct{Ap} is an array
of size $\ct{num\_var}+1$ and \ct{Ai} and \ct{Ax} are arrays of size equal to
the number of entries in $A$. They must be set such that
\ct{Ai[Ap[j]..Ap[j+1]-1]} and \ct{Ax[Ap[j]..Ap[j+1]-1]} hold the row indices and
nonzero values of the entries in column \ct{j} of $A$. The entries within each
column can be in any order, but there must be no duplicates. The input is
checked prior to running the solver and an error code is returned if it is
invalid (see Section~\ref{sec:info}).

\subsection{C Interface}
C wrapper functions for the methods of class \ct{LpSolver} are provided to allow
using the full functionality of IPX from C code. User code written in C must
include the header file \ct{include/ipx\_c.h}.

The following code snippet illustrates the use of the C interface. The
complete example program can be found in \ct{example/afiro\_c.c} and can be
compiled by calling \ct{make} in the \ct{example/} directory.

\begin{verbatim}
#include <math.h>
#include <stdio.h>
#include "ipx_c.h"

typedef ipxint Int;

#define NUM_VAR 12
#define NUM_CONSTR 9
const double obj[] = { -0.2194, 0.0, 0.0, 0.0, 0.0, 0.0, 0.0, 0.0, -0.32,
                       -0.5564, 0.6, -0.48 };
const double lb[NUM_VAR] = { 0.0 };
const double ub[] = { 80.0, 283.303, 283.303, 312.813, 349.187, INFINITY,
                      INFINITY, INFINITY, 57.201, 500.0, 500.501, 357.501};
// Constraint matrix in CSC format with 0-based indexing.
const Int Ap[] = { 0, 2, 6, 10, 14, 18, 20, 22, 24, 26, 28, 30, 32 };
const Int Ai[] = { 0, 5, ... };
const double Ax[] = { -1.0, 0.301, ... };
const double rhs[] = { 0.0, 80.0, 0.0, 0.0, 0.0, 0.0, 0.0, 44.0, 300.0 };
const char constr_type[] = { '<', '<', '=', '<', '<', '=', '<', '<', '<' };

int main() {
    void *lps = NULL;

    // Create new solver instance. This allocates a tiny amount of memory.
    ipx_new(&lps);
    if (!lps) return 1;

    // Solve the LP.
    Int status = ipx_solve(lps, NUM_VAR, obj, lb, ub, NUM_CONSTR, Ap, Ai, Ax,
                           rhs, constr_type);
    if (status != IPX_STATUS_solved) {
        // no solution
        // (invalid input, time/iter limit, numerical failure, out of memory)
        struct ipx_info info = ipx_get_info(lps);
        printf(" status: %ld, errflag: %ld\n", (long) status,
               (long) info.errflag);
        return 2;
    }

    // Get solver and solution information.
    struct ipx_info info = ipx_get_info(lps);

    // Get the interior solution (available if IPM was started).
    double x[NUM_VAR], xl[NUM_VAR], xu[NUM_VAR], slack[NUM_CONSTR];
    double y[NUM_CONSTR], zl[NUM_VAR], zu[NUM_VAR];
    ipx_get_interior_solution(lps, x, xl, xu, slack, y, zl, zu);

    // Get the basic solution (available if crossover terminated without error).
    double xbasic[NUM_VAR], sbasic[NUM_CONSTR];
    double ybasic[NUM_CONSTR], zbasic[NUM_VAR];
    Int cbasis[NUM_CONSTR], vbasis[NUM_VAR];
    ipx_get_basic_solution(lps, xbasic, sbasic, ybasic, zbasic, cbasis, vbasis);

    // Must call ipx_free() to deallocate memory in solver instance.
    ipx_free(&lps);
    return 0;
}
\end{verbatim}

The solver is accessed through a \ct{void} pointer, which must be allocated by
\ct{ipx\_new} and finally deallocated by \ct{ipx\_free} (see
\ct{include/ipx\_c.h} for documentation of these functions). The remaining
functions of the C interface are simply wrappers around the methods of class
\ct{LpSolver}, which require the pointer to the solver as their first argument.

Parameters and solver information are passed in and out through data types
\ct{struct ipx\_parameters} and \ct{struct ipx\_info}, which are the same as
\ct{ipx::Parameters} and \ct{ipx::Info} in the C++ interface and are documented
in Sections \ref{sec:parameters} and \ref{sec:info}. In contrast to the C++
interface, the struct members are uninitialized when a data type is defined.
A \ct{struct ipx\_parameters} object with default parameter values is returned
by \ct{ipx\_default\_parameters()}.

%-------------------------------------------------------------------------------
\section{Parameters}
\label{sec:parameters}
%-------------------------------------------------------------------------------

\param{display}{ipxint}{1}{
  If nonzero, then solver log is printed to standard output.
}

\param{logfile}{const char*}{\NULL}{
  Name of file to which solver log is appended. The file is created if it does
  not exist. If \ct{logfile} is \NULL\ or the empty string (\ct{""}), file
  logging is turned off.
}

\param{print\_interval}{double}{5.0}{
  Frequency (in seconds) for printing progress reports during construction of
  the starting basis and crossover. If negative, progress reports are turned
  off. If zero, a progress line is printed after each basis update.
}

\param{time\_limit}{double}{-1.0}{
  Time limit (in seconds) for the solver. If negative, no time limit is
  imposed.
}

\param{dualize}{ipxint}{-1}{
  Controls dualization of the LP model by the preprocessor. \\
  \begin{tabular}{rl}
    0 & model is not dualized \\
    $\ge1$ & model is dualized \\
    $<0$ & an automatic choice is made
  \end{tabular}
}

\param{scale}{ipxint}{1}{
  Controls the automatic scaling of the LP model by the preprocessor. \\
  \begin{tabular}{rl}
    $\le0$ & no scaling is applied \\
    1 & recursive equilibration of rows and columns of the constraint matrix \\
    $>1$ & currently the same as 1, but reserved for future options
  \end{tabular}
}

\param{ipm\_maxiter}{ipxint}{300}{
  Maximum number of interior point iterations.
}

\param{ipm\_feasibility\_tol}{double}{1e-6}{
  The interior point solver terminates when a feasibility and an optimality
  criterion are satisfied. The feasibility criterion requires that the relative
  primal and dual residuals are not larger than \ct{ipm\_feasibility\_tol}.
}

\param{ipm\_optimality\_tol}{double}{1e-8}{
  The interior point solver terminates when a feasibility and an optimality
  criterion are satisfied. The optimality criterion requires that the relative
  gap between the primal and dual objective values is not larger than
  \ct{ipm\_optimality\_tol}.
}

\param{ipm\_drop\_primal}{double}{1e-9}{
  Controls handling of primal degeneracies in the interior point solve. If a
  degenerate variable $\xl[j]$ or $\xu[j]$ is not larger than
  \ct{ipm\_drop\_primal}, then its dual variable is fixed at its current value
  and eliminated from the optimization; $\x[j]$ will then be at its bound in the
  solution. If \ct{ipm\_drop\_primal} is zero or negative, no primal variables
  are dropped. If the model was dualized by the preprocessor, then this option
  affects dual degeneracies in the input model.
}

\param{ipm\_drop\_dual}{double}{1e-9}{
  Controls handling of dual degeneracies in the interior point solve. If the
  dual variables $\zl[j]$ and $\zu[j]$ are degenerate and not larger than
  \ct{ipm\_drop\_dual}, then $\x[j]$ is fixed at its current value and
  eliminated from the optimization. The dual variables will then be zero in the
  solution. If \ct{ipm\_drop\_dual} is zero or negative, no dual variables are
  dropped. If the model was dualized by the preprocessor, then this option
  affects primal degeneracies in the input model.
}

\param{kkt\_tol}{double}{0.3}{
  Controls the accuracy to which the KKT linear equation systems are solved by
  an iterative method. A smaller value reduces the number of interior point
  iterations but increases the computational cost per iteration. Typical values
  are within the interval $[0.05,0.5]$.
}

\param{crash\_basis}{ipxint}{1}{
  Controls the construction of the starting basis for the preconditioner. \\
  \begin{tabular}{rl}
    $\le0$ & slack basis \\
    1 & crash method that prefers variables with a larger interior point scaling
    factor \\
    $>1$ & currently the same 1, but reserved for future options
  \end{tabular} \\
  The chosen procedure (slack basis, crash) is followed by a sequence of basis
  updates that makes free variables basic and fixed (slack) variables nonbasic.
}

\param{dependency\_tol}{double}{1e-6}{
  Controls the detection of linearly dependent rows and columns while
  constructing the starting basis. If possible, columns corresponding to free
  variables are pivoted into the basis, and slack columns corresponding to
  equality constraints are pivoted out of the basis. Hereby a nonbasic variable
  cannot replace a basic variable if the pivot element is less than or equal to
  \ct{dependency\_tol}. A negative value is treated as 0.0.
}

\param{volume\_tol}{double}{2.0}{
  Controls the update of the basis matrix from one interior point iteration to
  the next. An entry of the scaled tableau matrix is used as pivot element if it
  is larger than \ct{volume\_tol} in absolute value. Increasing the parameter
  usually leads to fewer basis updates but more iterations of the linear solver.
  Typical values are in the interval $[1.1,10.0]$. A value smaller than 1.0 is
  treated as 1.0.
}

\param{rows\_per\_slice}{ipxint}{10000}{
  Controls the update of the basis matrix from one interior point iteration to
  the next. The search for pivot elements partitions the tableau matrix into
  slices, each slice containing approximately \ct{rows\_per\_slice} rows. A
  smaller value leads a finer pivot search and possibly a better preconditioner,
  but makes the update procedure more expensive.
}

\param{maxskip\_updates}{ipxint}{10}{
  Controls the update of the basis matrix from one interior point iteration to
  the next. For each slice of the tableau matrix, the update search is
  terminated after computing \ct{maxskip\_updates} columns of the tableau matrix
  that do not contain eligible pivot elements. Decreasing the parameter makes
  the update procedure faster, but may affect the quality of the basis.
}

\param{lu\_kernel}{ipxint}{0}{
  Chooses the method/package for computing and updating the $LU$ factorization
  of basis matrices.\\
  \begin{tabular}{rl}
    $\le0$ & BASICLU for factorization and update \\
       1 & BASICLU for factorization, Forrest-Tomlin update without
       hypersparsity \\
       $>1$ & currently the same as 1, but reserved for future options
  \end{tabular}
}

\param{lu\_pivottol}{double}{0.0625}{
  Partial pivoting tolerance for the LU factorization. The tolerance is
  tightened automatically if a factorization is detected unstable.
}

\param{crossover}{ipxint}{1}{
  If nonzero, crossover is used for recovering an optimal basis.
}

\param{crossover\_start}{double}{1e-8}{
  Tightens the IPM termination criterion for crossover. At the beginning of
  crossover, the final IPM iterate is dropped to complementarity (i.\,e.\ for
  each pair of variables either the primal is set to its bound or the dual is
  set to zero). In addition to the standard IPM termination criterion, it is
  required that the relative primal or dual residual caused by dropping any
  variable is not larger than \ct{crossover\_start}. A nonpositive value means
  that the standard criterion is used. This parameter has no effect if crossover
  is turned off.
}

\param{pfeasibility\_tol}{double}{1e-7}{
  A basic solution is considered primal feasible if the primal variables do not
  violate their bounds by more than \ct{pfeasibility\_tol}.
}

\param{dfeasibility\_tol}{double}{1e-7}{
  A basic solution is considered dual feasible if the dual variables do not
  violate their sign condition by more than \ct{dfeasibility\_tol}.
}

%-------------------------------------------------------------------------------
\section{Info}
\label{sec:info}
%-------------------------------------------------------------------------------

Class \ct{ipx::Info} (for C++ users) and struct \ct{ipx\_info} (for C users)
provide information about the input problem, the solver operations and the
computed solution. Most of their members are almost self-explanatory and can be
found in \ct{include/ipx\_info.h}.

There are four members holding status and error codes that are documented below.
Possible values are macro-defined constants that have the prefix
\ct{IPX\_STATUS\_} and \ct{IPX\_ERROR\_}. They are defined in
\ct{include/ipx\_status.h}.\\[2em]

\info{status}{
  Termination status of the solver. Possible values are:\\[1em]
  \begin{tabular}{lp{8cm}}
    \ct{IPX\_STATUS\_not\_run} & Solver has not been called.\\
    \ct{IPX\_STATUS\_solved} & Solver terminated successfully. The solution
    status (optimal, imprecise, infeasible) is given by \textbf{status\_ipm} and
    \textbf{status\_crossover}.\\
    \ct{IPX\_STATUS\_stopped} & Solver stopped without having reached its
    termination criterion (time/iteration limit, numerical failure).
    \textbf{status\_ipm} or \textbf{status\_crossover} hold further
    information.\\
    \ct{IPX\_STATUS\_invalid\_input} & Solver was not started because the input
    to \ct{Solve()} was invalid. \textbf{errflag} holds further information.\\
    \ct{IPX\_STATUS\_out\_of\_memory} & Memory allocation failed.\\
    \ct{IPX\_STATUS\_internal\_error} & An internal error occured. This is a
    bug in IPX.\\
  \end{tabular}
}

\info{status\_ipm}{
  Status of the interior solution. Possible values are:\\[1em]
  \begin{tabular}{lp{8cm}}
    \ct{IPX\_STATUS\_not\_run} & IPM has not been started.\\
    \ct{IPX\_STATUS\_optimal} & IPM terminated successfully and the computed
    solution satifies the feasibility and optimality tolerances.\\
    \ct{IPX\_STATUS\_imprecise} & IPM reached its termination criterion, but the
    solution after postprocessing does not satisfy the feasibility or optimality
    tolerance.\\
    \ct{IPX\_STATUS\_primal\_infeas} & The LP problem is primal infeasible
    because equality constraints are inconsistent.\\
    \ct{IPX\_STATUS\_dual\_infeas} & The LP problem is dual infeasible because
    columns of the constraint matrix corresponding to free variables are
    linearly dependent and their objective coefficients cause an unbounded ray
    in primal space.\\
    \ct{IPX\_STATUS\_time\_limit} & IPM stopped because of time limit.\\
    \ct{IPX\_STATUS\_iter\_limit} & IPM stopped because of iteration limit.\\
    \ct{IPX\_STATUS\_no\_progress} & IPM stopped because no progress was
    achieved over a number of iterations.\\
    \ct{IPX\_STATUS\_failed} & Linear system solve or building the
    preconditioner failed. \textbf{errflag} holds further information.\\
  \end{tabular}
}

\info{status\_crossover}{
  Status of the basic solution obtained by crossover. Possible values
  are:\\[1em]
  \begin{tabular}{lp{8cm}}
    \ct{IPX\_STATUS\_not\_run} & Crossover has not been started.\\
    \ct{IPX\_STATUS\_optimal} & Crossover terminated successfully and the
    computed basic solution satifies the feasibility and optimality
    tolerances.\\
    \ct{IPX\_STATUS\_imprecise} & Crossover terminated successfully, but the
    basic solution does not satisfy the feasibility or optimality tolerance.
    Reoptimization with a simplex solver is needed.\\
    \ct{IPX\_STATUS\_time\_limit} & Crossover stopped because of time limit.\\
    \ct{IPX\_STATUS\_failed} & Crossover failed due to linear algebra issues.
    \textbf{errflag} holds further information.\\
  \end{tabular}
}

\info{errflag}{
  Possible values when \textbf{status} $=$
  \ct{IPX\_STATUS\_invalid\_input}:\\[1em]
  \begin{tabular}{lp{8cm}}
    \ct{IPX\_ERROR\_argument\_null} & A pointer argument was \ct{NULL}.\\
    \ct{IPX\_ERROR\_invalid\_dimension} & $\ct{num\_constr}<0$ or
    $\ct{num\_var}\le0$.\\
    \ct{IPX\_ERROR\_invalid\_matrix} & Constraint matrix was invalid
    ($\ct{Ap}[0]\neq0$, $\ct{Ap}[j]>\ct{Ap}[j+1]$ for some $j$, non-finite entry
    in \ct{Ax}, out-of-range or duplicate index in \ct{Ai}).\\
    \ct{IPX\_ERROR\_invalid\_vector} & One of the vectors \ct{obj}, \ct{lb},
    \ct{ub}, \ct{rhs} and \ct{constr\_type} contained an invalid entry, or
    $\ct{lb}[j]>\ct{ub}[j]$ for some $j$.\\
  \end{tabular}\\[1.5em]

  Possible values when \textbf{status\_ipm} or \textbf{status\_crossover} $=$
  \ct{IPX\_STATUS\_failed}:\\[1em]
  \begin{tabular}{lp{8cm}}
    \ct{IPX\_ERROR\_cr\_iter\_limit} &
    \multirow{5}{8cm}{The iterative method for solving the linear systems in the
      IPM did not converge.}\\
    \ct{IPX\_ERROR\_cr\_matrix\_not\_posdef} &\\
    \ct{IPX\_ERROR\_cr\_precond\_not\_posdef} &\\
    \ct{IPX\_ERROR\_cr\_no\_progress} &\\
    \ct{IPX\_ERROR\_cr\_inf\_or\_nan} &\\[1em]
    \ct{IPX\_ERROR\_basis\_singular} &
    \multirow{6}{8cm}{Factorizing or updating the basis matrix failed.}\\
    \ct{IPX\_ERROR\_basis\_almost\_singular} &\\
    \ct{IPX\_ERROR\_basis\_update\_singular} &\\
    \ct{IPX\_ERROR\_basis\_repair\_overflow} &\\
    \ct{IPX\_ERROR\_basis\_repair\_search} &\\
    \ct{IPX\_ERROR\_basis\_too\_ill\_conditioned}
  \end{tabular}\\[1em]
  In all cases the failure is caused by numerical difficulties. The specific
  error codes are meant for developers.
}

\end{document}
